\section{Expected Outcomes}

The thesis is expected to deliver the following outcomes:

\subsection{Technical contributions}

\begin{itemize}
  \item A working prototype that integrates schema change detection with
  LLM-based mapping regeneration in a single ETL-oriented workflow. The
  prototype will be documented and reproducible using the \texttt{etl-ai-schema}
  repository.
  \item Empirical evidence on the reliability of the detection module for the
  schema change types considered (additions, removals, renames, type changes),
  including precision and recall where applicable.
  \item Empirical evidence on the extent to which the RAG-grounded LLM produces
  executable and semantically correct transformation updates for the NYC taxi
  V1$\rightarrow$V2 and V2$\rightarrow$V3 scenarios, including qualitative
  notes on necessary manual fixes.
\end{itemize}

\subsection{Research contributions}

\begin{itemize}
  \item A comparative evaluation of manual vs.\ AI-assisted adaptation in
  terms of time to successful pipeline run and correctness of outputs. The
  expectation is that the AI-assisted workflow will reduce adaptation time for
  the evaluated scenarios, while possibly requiring some manual intervention
  (e.g., minor code fixes or validation).
  \item A discussion of limitations and failure modes of the current approach
  (e.g., sensitivity to prompt design, LLM non-determinism, and complex
  structural changes), providing guidance for future work.
\end{itemize}

\subsection{Practical impact}

The thesis aims to demonstrate that an integrated detection-and-regeneration
workflow is feasible and can reduce the manual effort required to keep ETL
pipelines operational under schema evolution. The results are intended to
inform practitioners and researchers about the potential and limits of
retrieval-augmented LLMs for ETL maintenance.
