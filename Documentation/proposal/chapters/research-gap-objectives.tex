\section{Research Gap}

The reviewed literature and initial implementation work reveal several gaps
that motivate the thesis:
%
\begin{itemize}
  \item \textbf{End-to-end workflows.} Classical mapping adaptation frameworks
  typically start from an explicit description of schema changes and do not
  address the problem of detecting evolution directly from evolving datasets
  and operational artefacts \cite{velegrakis2003mapping,yu2005semantic}.
  \item \textbf{RAG for mapping regeneration.} While RAG has been applied to
  knowledge-intensive NLP tasks \cite{lewis2020retrieval} and explored for
  schema matching \cite{sheetrit2024rematch}, there is limited work on using
  retrieval-augmented LLMs to regenerate executable ETL transformation code
  grounded in schema and change descriptions.
  \item \textbf{Operational integration and evaluation.} Many proposals are
  evaluated on idealized benchmarks and do not consider integration into
  orchestrated ETL workflows with realistic datasets, monitoring, and
  rollback mechanisms.
\end{itemize}

\section{Research Aim and Objectives}

The overall aim of the thesis is:
%
\begin{quote}
  \emph{To design and evaluate an adaptive approach for schema evolution
  detection and mapping regeneration in ETL pipelines, leveraging
  retrieval-augmented large language models to produce executable, semantically
  grounded transformation updates.}
\end{quote}

To achieve this aim, the thesis pursues the following objectives:
%
\begin{enumerate}
  \item \textbf{Design a schema evolution detection mechanism} that identifies
  and classifies schema changes relevant to ETL transformations (e.g.,
  additions, deletions, renames, type changes).
  \item \textbf{Develop a retrieval strategy} that collects relevant artefacts
  (historical schemas, mapping specifications, transformation scripts, and
  example instances) to ground LLM-based regeneration.
  \item \textbf{Implement an LLM-assisted mapping regeneration component} that
  generates candidate transformation updates informed by retrieved evidence.
  \item \textbf{Evaluate the approach} in terms of detection accuracy, mapping
  correctness, and operational performance, comparing manual and AI-assisted
  adaptation workflows.
\end{enumerate}

\section{Research Questions}

The thesis is guided by the following research questions:
%
\begin{itemize}
  \item \textbf{RQ1 (Detection):} Which schema change types can be detected and
  classified reliably from evolving tabular datasets and associated metadata in
  realistic ETL scenarios?
  \item \textbf{RQ2 (Regeneration):} To what extent can RAG-grounded LLMs
  generate executable transformation updates that preserve the intended
  semantics of existing ETL mappings?
  \item \textbf{RQ3 (Operational feasibility):} What are the runtime and
  maintenance trade-offs of integrating detection and regeneration into an
  orchestrated ETL workflow, compared to a purely manual adaptation process?
\end{itemize}

These questions are designed to be empirically answerable using controlled
schema evolution scenarios for both synthetic and real-world datasets, as
described in the data and methodology chapters.

