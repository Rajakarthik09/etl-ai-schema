\section{Overview of Data Sources}

Two families of datasets are used in this project: a synthetic users dataset
for rapid prototyping and unit-level validation, and a real-world NYC yellow
taxi dataset for the main evaluation. Both are represented as multiple schema
versions to simulate controlled evolution.

\section{Synthetic Users Dataset}

The synthetic users dataset is generated by dedicated scripts in the
repository. It consists of three versions (\texttt{users\_v1.csv},
\texttt{users\_v2.csv}, and \texttt{users\_v3.csv}), each containing a small
table of user profiles with deliberate schema changes. Examples of changes
include:
%
\begin{itemize}
  \item adding or removing attributes (e.g., contact details, subscription
  information),
  \item renaming existing columns (e.g., \texttt{age} to \texttt{user\_age}),
  and
  \item changing the representation of timestamps (e.g., to ISO~8601).
\end{itemize}

This dataset is used primarily for early testing of the schema detection logic
and the basic behaviour of the mapping regenerator. All substantive evaluation
of the proposed approach is based on the NYC yellow taxi dataset described
below.

\section{NYC Yellow Taxi Dataset}

For the main evaluation, the project uses real trip records from the NYC Taxi
and Limousine Commission (TLC). The source data are publicly available
Parquet files (\texttt{yellow\_tripdata\_YYYY-MM.parquet}) from the official
TLC trip record data portal. A conversion script loads a selected Parquet file
(e.g., \texttt{yellow\_tripdata\_2015-01.parquet}) and writes a down-sampled
CSV (approximately 100{,}000 rows) to \texttt{data/raw/yellow\_base\_v1.csv},
which defines the baseline taxi schema (V1).

Controlled schema evolution is then applied via scripts that construct V2 and
V3 from V1:
%
\begin{itemize}
  \item \textbf{V1} includes canonical TLC fields such as \texttt{VendorID},
  \texttt{tpep\_pickup\_datetime}, \texttt{tpep\_dropoff\_datetime},
  \texttt{passenger\_count}, \texttt{trip\_distance}, \texttt{RatecodeID},
  \texttt{store\_and\_fwd\_flag}, \texttt{PULocationID}, \texttt{DOLocationID},
  \texttt{payment\_type}, \texttt{fare\_amount}, \texttt{tip\_amount},
  \texttt{tolls\_amount}, \texttt{total\_amount}, and related monetary fields.
  \item \textbf{V2} applies moderate changes: renaming \texttt{trip\_distance}
  to \texttt{trip\_km}, adding a derived column \texttt{tip\_ratio}, dropping
  \texttt{payment\_type}, and casting \texttt{VendorID} from integer to string.
  \item \textbf{V3} introduces further changes: renaming \texttt{trip\_km} to
  \texttt{trip\_distance\_km}, adding a compound field \texttt{pickup\_info}, and
  adding a categorical column \texttt{time\_of\_day} derived from the pickup
  hour.
\end{itemize}

The resulting files (\texttt{yellow\_base\_v1.csv}, \texttt{yellow\_base\_v2.csv},
and \texttt{yellow\_base\_v3.csv}) form a controlled yet realistic schema
evolution sequence grounded in genuine taxi data. Schema and change descriptions
for all V1/V2/V3 combinations are generated and persisted under
\texttt{schemas/} for use by the detection and regeneration modules.
