\begin{abstract}
\noindent Modern data platforms depend on Extract--Transform--Load (ETL) pipelines to integrate heterogeneous data sources into consistent analytical representations. In practice, upstream schemas evolve frequently due to changing business requirements, software updates, and incremental feature development. These schema changes can invalidate transformation logic and schema mappings, causing pipeline failures and expensive manual maintenance. Prior research has developed foundational approaches for mapping adaptation under schema evolution, including incremental rewriting of mappings and composition-based adaptation, but these approaches typically assume that schema changes are known and rely on rule-based techniques that do not directly address the automation of transformation regeneration in operational environments \cite{velegrakis2003mapping,yu2005semantic}.

\noindent This thesis investigates an adaptive approach to schema evolution in ETL pipelines that combines automated change detection with retrieval-augmented large language models (RAG-LLMs). The central idea is to ground mapping regeneration in retrieved evidence such as historical schema versions, prior validated mappings, transformation code, and example instances, thereby improving the correctness and reliability of generated updates \cite{lewis2020retrieval,chen2021evaluating}. The work proposes (i) a schema change detection and classification mechanism, (ii) a retrieval strategy to assemble task-relevant context, and (iii) an LLM-assisted mapping regeneration component that produces executable transformation updates. The approach is evaluated using reproducible schema evolution scenarios and public datasets, measuring detection accuracy, mapping correctness, and operational performance. The expected contribution is a practical, evidence-grounded workflow for reducing manual ETL maintenance effort while preserving the intended semantics of existing mappings.
\end{abstract}